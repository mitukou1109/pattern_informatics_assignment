\documentclass[a4paper,12pt]{jsarticle}
\usepackage{amsmath,amssymb,amsthm}
\usepackage{bm}
\usepackage{booktabs}
\usepackage{caption}
\usepackage[margin=20truemm]{geometry}
\usepackage[dvipdfmx]{graphicx}
\usepackage{multirow}
\usepackage{url}
\usepackage[subrefformat=parens]{subcaption}
\usepackage{wrapfig}

\renewcommand{\figurename}{Fig. }
\renewcommand{\tablename}{Table }

\allowdisplaybreaks[4]

\title{パターン情報学 プログラミング課題}
\author{03-223008 坂本光皓}
\date{\today}

\begin{document}

\maketitle

\section{課題1}
線形・二次・三次の識別関数を表\ref{table:1_parameter}に示すパラメータを用いて学習させ,得られた決定境界を図\ref{fig:1_decision_boundary}に示す.いずれもクラス間を適当に分離していることがわかる.

\begin{table}[htbp]
  \centering
  \caption{学習に用いたパラメータ}
  \label{table:1_parameter}
  \begin{tabular}{cccc}
    \toprule
    項目            & 線形    & 二次    & 三次    \\
    \midrule
    学習係数          & 0.001 & 0.001 & 0.002 \\
    エポック数         & 100   & 100   & 100   \\
    ガウシアンノイズの標準偏差 & -     & 0.1   & 0.3   \\
    \bottomrule
  \end{tabular}
\end{table}

\begin{figure}[htbp]
  \centering
  \begin{minipage}[b]{0.325\linewidth}
    \centering
    \includegraphics[width=\linewidth]{img/1_decision_boundary_linear.png}
    \subcaption{線形識別関数}
    \label{fig:1_decision_boundary_linear}
  \end{minipage}
  \begin{minipage}[b]{0.325\linewidth}
    \centering
    \includegraphics[width=\linewidth]{img/1_decision_boundary_quadratic.png}
    \subcaption{二次識別関数}
    \label{fig:1_decision_boundary_quadratic}
  \end{minipage}
  \begin{minipage}[b]{0.325\linewidth}
    \centering
    \includegraphics[width=\linewidth]{img/1_decision_boundary_cubic.png}
    \subcaption{三次識別関数}
    \label{fig:1_decision_boundary_cubic}
  \end{minipage}
  \caption{パーセプトロンで学習した決定境界}
  \label{fig:1_decision_boundary}
\end{figure}

次に,学習係数を0.2,0.002,0.0002とした場合の決定境界を図\ref{fig:1_decision_boundary_learning_rate}に示す.

\begin{figure}[htbp]
  \centering
  \begin{minipage}[b]{0.325\linewidth}
    \centering
    \includegraphics[width=\linewidth]{img/1_decision_boundary_0_2.png}
    \subcaption{0.2の場合}
    \label{fig:1_decision_boundary_0_2}
  \end{minipage}
  \begin{minipage}[b]{0.325\linewidth}
    \centering
    \includegraphics[width=\linewidth]{img/1_decision_boundary_cubic.png}
    \subcaption{0.002の場合}
    \label{fig:1_decision_boundary_0_002}
  \end{minipage}
  \begin{minipage}[b]{0.325\linewidth}
    \centering
    \includegraphics[width=\linewidth]{img/1_decision_boundary_0_0002.png}
    \subcaption{0.0002の場合}
    \label{fig:1_decision_boundary_0_0002}
  \end{minipage}
  \caption{学習係数による決定境界の比較}
  \label{fig:1_decision_boundary_learning_rate}
\end{figure}

\end{document}