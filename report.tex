\documentclass[a4paper,12pt]{jsarticle}
\usepackage{amsmath,amssymb,amsthm}
\usepackage{bm}
\usepackage{booktabs}
\usepackage{caption}
\usepackage[margin=15truemm]{geometry}
\usepackage[dvipdfmx]{graphicx}
\usepackage{multirow}
\usepackage{url}
\usepackage[subrefformat=parens]{subcaption}
\usepackage{wrapfig}

\makeatletter
\newcommand{\figcap}{\def\@captype{figure}}
\newcommand{\tabcap}{\def\@captype{table}}
\makeatother

\renewcommand{\figurename}{Fig. }
\renewcommand{\tablename}{Table }

\allowdisplaybreaks[4]

\title{パターン情報学 プログラミング課題}
\author{03-223008 坂本光皓}
\date{\today}

\begin{document}

\maketitle

\subsection*{課題1}

線形・二次・三次の識別関数を\tablename\ref{table:1_parameter_dataset}に示すパラメータおよびデータセットを用いてパーセプトロンで学習させ,得られた決定境界を\figurename\ref{fig:1_decision_boundary}に示す.いずれもクラス間を適当に分離していることがわかる.

\begin{table}[htbp]
  \centering
  \caption{学習に用いたパラメータおよびデータセット}
  \label{table:1_parameter_dataset}
  \begin{tabular}{cccc}
    \toprule
                             & 線形                   & 二次            & 三次          \\
    \midrule
    学習係数                     & 0.001                & 0.001         & 0.002       \\
    エポック数                    & 100                  & 100           & 100         \\
    \midrule
    データセット(sklearn.datasets) & make\_classification & make\_circles & make\_moons \\
    ガウシアンノイズの標準偏差            & -                    & 0.1           & 0.3         \\
    \bottomrule
  \end{tabular}
\end{table}

\begin{figure}[htbp]
  \centering
  \begin{minipage}[b]{0.25\linewidth}
    \centering
    \includegraphics[width=\linewidth]{img/1_decision_boundary_linear.png}
    \subcaption{線形識別関数}
    \label{fig:1_decision_boundary_linear}
  \end{minipage}
  \begin{minipage}[b]{0.25\linewidth}
    \centering
    \includegraphics[width=\linewidth]{img/1_decision_boundary_quadratic.png}
    \subcaption{二次識別関数}
    \label{fig:1_decision_boundary_quadratic}
  \end{minipage}
  \begin{minipage}[b]{0.25\linewidth}
    \centering
    \includegraphics[width=\linewidth]{img/1_decision_boundary_cubic.png}
    \subcaption{三次識別関数}
    \label{fig:1_decision_boundary_cubic}
  \end{minipage}
  \caption{パーセプトロンで学習した決定境界}
  \label{fig:1_decision_boundary}
\end{figure}

次に,三次の識別関数の学習において学習係数を0.2,0.002,0.0002,その他のパラメータを\tablename\ref{table:1_parameter_dataset}に示す値とした場合の決定境界を\figurename\ref{fig:1_decision_boundary_learning_rate}に示す.\subref{fig:1_decision_boundary_lr_0_2}では\subref{fig:1_decision_boundary_lr_0_002}に比べてよりデータの分布に沿った決定境界が得られている一方,\subref{fig:1_decision_boundary_lr_0_0002}では学習が十分に進んでおらず,クラス間を正しく分離できていないことがわかる.

\begin{figure}[htbp]
  \centering
  \begin{minipage}[b]{0.25\linewidth}
    \centering
    \includegraphics[width=\linewidth]{img/1_decision_boundary_lr_0_2.png}
    \subcaption{0.2の場合}
    \label{fig:1_decision_boundary_lr_0_2}
  \end{minipage}
  \begin{minipage}[b]{0.25\linewidth}
    \centering
    \includegraphics[width=\linewidth]{img/1_decision_boundary_cubic.png}
    \subcaption{0.002の場合}
    \label{fig:1_decision_boundary_lr_0_002}
  \end{minipage}
  \begin{minipage}[b]{0.25\linewidth}
    \centering
    \includegraphics[width=\linewidth]{img/1_decision_boundary_lr_0_0002.png}
    \subcaption{0.0002の場合}
    \label{fig:1_decision_boundary_lr_0_0002}
  \end{minipage}
  \caption{学習係数による決定境界の比較}
  \label{fig:1_decision_boundary_learning_rate}
\end{figure}

さらに,エポック数を10,100,1000,その他のパラメータを\tablename\ref{table:1_parameter_dataset}の値とした場合の決定境界を\figurename\ref{fig:1_decision_boundary_epoch}に示す.\subref{fig:1_decision_boundary_epoch_10}では学習不足のため識別の誤りが多くみられるほか,\subref{fig:1_decision_boundary_epoch_1000}では学習率0.2の場合と同様に決定境界がより明瞭になっていることがわかる.

\begin{figure}[htbp]
  \centering
  \begin{minipage}[b]{0.25\linewidth}
    \centering
    \includegraphics[width=\linewidth]{img/1_decision_boundary_epoch_10.png}
    \subcaption{10の場合}
    \label{fig:1_decision_boundary_epoch_10}
  \end{minipage}
  \begin{minipage}[b]{0.25\linewidth}
    \centering
    \includegraphics[width=\linewidth]{img/1_decision_boundary_cubic.png}
    \subcaption{100の場合}
    \label{fig:1_decision_boundary_epoch_100}
  \end{minipage}
  \begin{minipage}[b]{0.25\linewidth}
    \centering
    \includegraphics[width=\linewidth]{img/1_decision_boundary_epoch_1000.png}
    \subcaption{1000の場合}
    \label{fig:1_decision_boundary_epoch_1000}
  \end{minipage}
  \caption{エポック数による決定境界の比較}
  \label{fig:1_decision_boundary_epoch}
\end{figure}

最後に,ノイズの標準偏差を0.1,0.3,0.5,その他のパラメータを\tablename\ref{table:1_parameter_dataset}の値とした場合の決定境界を\figurename\ref{fig:1_decision_boundary_noise}に示す.\subref{fig:1_decision_boundary_noise_0_1}ではノイズが少ないためデータを完全に分離できており,19エポックで正解率が100\%に達したが,\subref{fig:1_decision_boundary_noise_0_5}では決定境界の形状がその他と大きく異なっており,正解率は80\%程度と\subref{fig:1_decision_boundary_noise_0_3}の約90\%に比べ低い結果となった.

\begin{figure}[htbp]
  \centering
  \begin{minipage}[b]{0.25\linewidth}
    \centering
    \includegraphics[width=\linewidth]{img/1_decision_boundary_noise_0_1.png}
    \subcaption{0.1の場合}
    \label{fig:1_decision_boundary_noise_0_1}
  \end{minipage}
  \begin{minipage}[b]{0.25\linewidth}
    \centering
    \includegraphics[width=\linewidth]{img/1_decision_boundary_cubic.png}
    \subcaption{0.3の場合}
    \label{fig:1_decision_boundary_noise_0_3}
  \end{minipage}
  \begin{minipage}[b]{0.25\linewidth}
    \centering
    \includegraphics[width=\linewidth]{img/1_decision_boundary_noise_0_5.png}
    \subcaption{0.5の場合}
    \label{fig:1_decision_boundary_noise_0_5}
  \end{minipage}
  \caption{ノイズの標準偏差による決定境界の比較}
  \label{fig:1_decision_boundary_noise}
\end{figure}

\subsection*{課題2}

\tablename\ref{table:2_parameter_dataset}に示すパラメータおよびデータセットを用いて学習させたロジスティック回帰モデルの性能を\tablename\ref{table:2_performance}に示す.なお,使用したデータセットは手書き数字の画像データであり,各ラベルは0から9までの数字に対応している.

3,8,9の再現率,適合率,F値が他のクラスに比べ低く,特に8の適合率に大きな差がみられることから,この3クラスは8と誤認識されやすいことがわかる.一方で正解率は0.971と高く,全体としては高い性能を示しているといえる.

\begin{table}[htbp]
  \centering
  \caption{学習に用いたパラメータおよびデータセット}
  \label{table:2_parameter_dataset}
  \begin{tabular}{cc}
    \toprule
    学習係数                    & 0.001                           \\
    エポック数                   & 1000                            \\
    \midrule
    \multirow{2}{*}{データセット} & sklearn.datasets.digit\_dataset \\
                            & (学習データ80\%,テストデータ20\%)          \\
    \bottomrule
  \end{tabular}
\end{table}

\begin{table}[htbp]
  \centering
  \caption{学習したモデルの性能}
  \label{table:2_performance}
  \begin{tabular}{ccccccccccc}
    \toprule
        & \multicolumn{10}{c}{ラベル}                                                                           \\
        & 0                          & 1     & 2     & 3     & 4     & 5     & 6     & 7     & 8     & 9     \\
    \midrule
    再現率 & 0.993                      & 0.962 & 0.988 & 0.965 & 0.972 & 0.975 & 0.976 & 0.980 & 0.935 & 0.958 \\
    適合率 & 0.993                      & 0.941 & 0.994 & 0.982 & 0.988 & 0.970 & 0.982 & 0.981 & 0.916 & 0.957 \\
    F値  & 0.993                      & 0.951 & 0.991 & 0.973 & 0.980 & 0.973 & 0.979 & 0.981 & 0.925 & 0.957 \\
    正解率 & \multicolumn{10}{c}{0.971}                                                                         \\
    \bottomrule
  \end{tabular}
\end{table}

\subsection*{課題3}
$k = 16$として\tablename\ref{table:1_parameter_dataset}に示すデータセットを用いて$k$近傍法で得られる決定境界を\figurename\ref{fig:3_decision_boundary}に示す.\figurename\ref{fig:1_decision_boundary}のパーセプトロンで得られたものと類似しているが,$k$近傍法のほうが凹凸が激しくノイズに適合していることがわかる.

\begin{figure}[htbp]
  \centering
  \begin{minipage}[b]{0.25\linewidth}
    \centering
    \includegraphics[width=\linewidth]{img/3_decision_boundary_classification.png}
    \subcaption{make\_classificationの場合}
    \label{fig:3_decision_boundary_classification}
  \end{minipage}
  \begin{minipage}[b]{0.25\linewidth}
    \centering
    \includegraphics[width=\linewidth]{img/3_decision_boundary_circles.png}
    \subcaption{make\_circlesの場合}
    \label{fig:3_decision_boundary_circles}
  \end{minipage}
  \begin{minipage}[b]{0.25\linewidth}
    \centering
    \includegraphics[width=\linewidth]{img/3_decision_boundary_moons.png}
    \subcaption{make\_moonsの場合}
    \label{fig:3_decision_boundary_moons}
  \end{minipage}
  \caption{$k$近傍法で得られる決定境界}
  \label{fig:3_decision_boundary}
\end{figure}

次に,\tablename\ref{table:3_parameter_dataset}に示すパラメータおよびデータセットを用いて,交差検証法により$k$の値に対する正解率の変化を調べた結果を\figurename\ref{fig:3_k_accuracy}に示す.多少のぶれがみられるものの,$k$が増加するにつれて正解率が低下しており,$k = 1$の場合に最大となった.したがって,今回のデータセットに対しては$k = 1$が最適な値であるといえる.

\begin{figure}[htbp]
  \centering
  \begin{minipage}[c]{0.5\linewidth}
    \centering
    \tabcap \caption{評価に用いたパラメータおよびデータセット}
    \label{table:3_parameter_dataset}
    \begin{tabular}{cc}
      \toprule
      $k$                     & 1~32                            \\
      \midrule
      \multirow{2}{*}{データセット} & sklearn.datasets.digit\_dataset \\
                              & (学習データ50\%,テストデータ50\%)          \\
      分割数                     & 100                             \\
      \bottomrule
    \end{tabular}
  \end{minipage}
  \hspace{0.05\linewidth}
  \begin{minipage}[c]{0.35\linewidth}
    \centering
    \includegraphics[width=\linewidth]{img/3_k_train_accuracy.png}
    \figcap \caption{$k$の値に対する正解率の変化}
    \label{fig:3_k_accuracy}
  \end{minipage}
\end{figure}

最後に,テストデータに対する正解率の変化を\figurename\ref{fig:3_k_test_accuracy}に示す.学習データと同様に,$k = 1$の場合に0.984と最も高くなった.

\begin{figure}[htbp]
  \centering
  \includegraphics[width=0.35\linewidth]{img/3_k_test_accuracy.png}
  \caption{テストデータに対する正解率の変化}
  \label{fig:3_k_test_accuracy}
\end{figure}

\subsection*{実装について}

ソースコードはPythonで記述し,数値計算にNumPy,データセットの用意にscikit-learn,グラフ描画にmatplotlibを用いた.プログラムの詳細ならびに実行方法は同梱したREADME.mdを参照されたい.

\end{document}